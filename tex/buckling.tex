\section{Dewar Buckling Analysis}
\label{sec:buckling}

The Dewar Buckling Analysis was performed by Cryofab and documented in 
App.~\ref{app:buckling}.
It indicates that the inner vessel will collapse if there is 4.98~psid between 
the jacket and the inner vessel. 
This presents an operational challenge and a risk that is not easily mitigated 
without a gauge on the jacket. 
Loss of the vacuum in the jacket is possible, but unlikely.\\

Loss of the vacuum in the vacuum jacket is a possible but rare event.
If the vacuum in the vacuum jacket of the dewar is lost during operation,
it will likely be observed as the required cooling power is increasing.
The outside wall of the dewar will be colder than room temperature,
possibly condensing water.
If the cooling power provided is lower than required,
the pressure in the inner vessel will rise and the argon gas will be released
from the back pressure regulator when the pressure reaches 4~psig.
If the pressure continues increasing, the burst disc will crack preventing
an explosion and preventing injuries or damages to the equipment.
The pressure rise in this case is expected to take a few hours,
ample for operators to react before cracking the burst disc,
unless the vacuum jacket is damaged and the vacuum is immediately lost.\\

If the vacuum in the jacket fails during the non-operating period,
it may not be detected.
If the dewar is evacuated, the inner vessel will buckle resulting in a damaged
and unusable dewar, but will not present a hazard to personnel.
This risk is classified as ``low'' even without a mitigation. 
The risk has been added to the ``Filling Tab'' of the ``SLArPAAS I Hazard Analysis''
(\url{https://docs.google.com/spreadsheets/d/1sxUV8Ua3J8l6XIYgdW33Ia-IjVwUAQS-_h5lZe0bloU/edit?usp=sharing}).
Also, the Operations Procedure/Checklist, included in the package, has be updated to warn 
regarding this potential.