\section{Instroduction}
\label{sec:intro}

This engineering note describes a cryogenic vessel's suitability for use in 
a SLAC LDRD project for liquid-argon time-projection chamber (LArTPC) research
and development,
seeking approval for its use at SLAC.\\

The Fundamental Physics Directorate (FPD) DUNE group at SLAC had developed 
and operated two liquid argon cryogenic systems for neutrino detector R\&D
in the Liquid Noble Test Facility (LNTF) at IR2 (building 620).
The Stage One system, LAIR2, consists of a 65-liter cryogenic vessel and
a high voltage test setup,
while the Stage Two system, SLArchetto, includes a 264-liter liquid argon
cryogenic vessel with a time-projection chamber (TPC).\\

The scientists in the DUNE and LZ groups are embarking on Stage Three,
SLArpaas I (SLAC (LAr) Prototype Ar Assembly System I),
which involves a 60-liter liquid argon cryogenic vessel and a TPC.
The lead physicist is Yun-Tse Tsai, while the lead physicist for the cryogenic
and piping system is Miriam Moore.
The ESH coordinator supporting them is Richard Delacruz.\\

The SLArpaas I main dewar is a commercially available item manufactured by Cryofab
in New Jersey.
This is a reputable manufacturer with a long history of providing cryogenic vessels 
for scientific research. 
However, the vessels manufactured by Cryofab do not normally come with an ASME stamp. 
This engineering note has been developed to demonstrate its suitability for use in 
this experiment and seek approval for its use at SLAC.\\

The top plate of the vessel was designed by SLAC scientist Brian Lenardo
and was manufactured by Kurt J. Lesker.\\

The supporting documents described in this Note are either directly included in 
the Appendices or distributed with this document in an electronic zip file.\\
