\section{Ring Flange Stress}
\label{app:ring_flange}

% SLArpaas has flat face flanges with metal-to-metal contact outside the bolt
% circle, and Figure~Y-5.1.3 in non-Mandatory Appendix~Y, Class~3 Flange Assembly
% applies.

The SLArPAAS I dewar and flange is made of stainless steel 304,
and the parameters are listed in Table~\ref{table:ring_flange}.
The calculations in this appendix can be found at
\url{https://github.com/yuntsebaryon/SLArpaasPressureAna/blob/main/notebook/RingFlangeStress.ipynb}.\\

%-------------------------------------------------------------------
\begin{table}[h]
\begin{center}
\tabcolsep=10pt
\begin{tabular}{r|c|l|l}
\hline
\hline
Variable & & Value & Unit \\
\hline
Outside diameter of flange & $A$ & 18 & in \\
Inside diameter of flange & $B$ & 14 & in \\
Bolt‐circle diameter & $C$ & 17 & in \\
Outside diameter of gasket contact face & $G$ & 15.086 & in \\
Flange thickness & $t$ & 0.63 & in \\
Number of bolts & $n$ & 8 & n/a \\
Internal design pressure & $P$ & 15 & psig \\
Nominal bolt diameter & $a$ & 0.313 & in \\
\hline
\hline
\end{tabular}
\caption{Flange parameters of the dewar based on the Cryofab drawing
and analysis.  Note that the flange thickness refers to the flange
of the dewar.  The nominal bolt diameter, $a$, is taken from 5/16"-18 bolts.}
\label{table:ring_flange}
\end{center}
\end{table}
%-------------------------------------------------------------------


% ----------------------------------------------------------------------
\subsection{Bolt Load}
\label{app:bolt_load}

SLArPAAS I has an O-ring, one of the self-energizing types (Table~2.5-1),
and therefore Mandatory Appendix~2 is referred herein.

According to App.~2-3, the total joint-contact surface compression load,
\begin{equation}
    H_P = 2b\times \pi GmP,
\end{equation}
where $b$ is effective gasket or joint‐contact‐surface seating width
and $m$ is gasket factor obtained from Table 2-5.1.
Table 2-5.1 shows that in our case, $m = 0$, and therefore $H_P = 0$.

The minimum required bolt load for the operating conditions,
$W_{m1}$, can be obtained following Eq.~2-5(c)(3)(-a) with $H_p = 0$ (see below),
\begin{equation}
    W_{m1} = H + H_p = H = 0.785G^2 P,
\end{equation}
where $H$ is total hydrostatic end force.
$W_{m1}$ is 2679.84~pound.
Adding the weights of the SLArPAAS I vessel, less than 60~pounds,
and of the liquid argon filling with the vessel, 84~kg or 185.19~pound
(60~L with the density 1.4~kg/L),
the minimum required bolt load for the operation conditions is 2924.64~pounds.\\

The minimum required bolt load for gasket seating, $W_{m2}$, according to
2-5(c)(3)(-b), is 0.

% ----------------------------------------------------------------------
\subsection{Bolt Spacing}
\label{app:bolt_spacing}

According to 2-5(d), the maximum bolt space, $B_{s,max}$, can be obtained
by Eq.~2.5(d)(3)

\begin{equation}
    B_{s,max} = 2a + \frac{6t}{m+0.5}.
\end{equation}

The cord length of the bolt spacing on SLArPAAS I can be calculated with
\begin{equation}
    B_s = \frac{C}{2}\frac{\sin(2\pi/n)}{\sin((\pi-(2\pi/n))/2)}.
\end{equation}

The maximum bolt spacing $B_{s,max}$ therefore should be 8.186~in,
while SLArPAAS I has the bolt spacing 6.5~in.
The requirement is satisfied.

% ----------------------------------------------------------------------
\subsection{Bolt Area}
\label{app:bolt_area}

The specified minimum tensile strength at room temperature, $S_T$,
is 70000~psi in the Cryofab analysis.
The bolts on McMaster are mostly made of SAE J429, and the Grade 2
(low or medium carbon steel) with 1/4" - 3/4" size has the minimum 
tensile strength 74000~psi.
Therefore $S_T = 70000$~psi is a reasonable number.\\

The allowable bolt stress at design temperature (see UG-23), $S_b$,
is defined as $S_b = S_T/4$ in the Cryofab analysis.
This can be found in Table 6-100(a), 6-100(b), 6-100(c) in Section~II 
Part~D of the ASME code.\\

The total cross‐sectional area of bolts at root of thread or section 
of least diameter under stress,
required for the operating conditions, $A_{m1} = W_{m1}/S_b$.\\

The minimum required bolt area, $A_{m1} = 0.153~\text{in}^2$,
while the area of each bolt, $A_b$, is 0.045~in$^2$, according to
the Cryofab analysis.
Note that the area of a 5/16"-18 bolt is 0.077~in$^2$.
The total bolt area at SLArPAAS I is $A_bn = 0.045\times 8 = 0.36~\text{in}^2$,
satisfying the requirement ($>0.153~\text{in}^2$).

% ----------------------------------------------------------------------
\subsection{Flange Design Bolt Load}
\label{app:flange_bolt_load}

According to Mandatory Appendix 2-5(e) in the ASME code,
the flange design bolt load, $W$, can be obtained by

\begin{equation}
    W = W_{m1}
\end{equation}

for operating conditions, while

\begin{equation}
    W = \frac{(A_m+A_b)S_a}{2}
\label{eq:flange_bolt_load_gasket_seating}
\end{equation}

for gasket seating.
$S_a$ is allowable bolt stress at atmospheric temperature (UG-23).
Also 2-5(e) explains,

\begin{displayquote}
    Sa used in eq. (5) (Eq.~\ref{eq:flange_bolt_load_gasket_seating} herein) 
    shall be not less than that tabulated in the stress tables (see UG-23). 
    In addition to the minimum requirements for safety, eq. (5) provides 
    a margin against abuse of the flange from overbolting. 
    Since the margin against such abuse is needed primarily for the initial, 
    bolting‐up operation which is done at atmospheric temperature and before 
    application of internal pressure, the flange design is required to satisfy 
    this loading only under such conditions.
\end{displayquote}

In this specific case, $S_a = S_b$, $A_m = A_{m1}$.\\

The flange design bolt load is 4612.32~pounds 
higher than the minimum required bolt load, 2924.64~pounds
obtained in App.~\ref{app:bolt_load}.

% ----------------------------------------------------------------------
\subsection{Flange Moment}
\label{app:flange_moment}

For the operating conditions, the total flange moment is described
in Appendix~2-6 in the ASME code,

\begin{equation}
    M_{op} = M_D+M_T+M_G,
\end{equation}

where $M_{op}$ is denoted as $M_o$ in the ASME code,
$M_D = H_Dh_D$, $M_T = H_Th_T$ and $M_G = H_Gh_G$.

All these variables can be found in Appendix~2-3 in the ASME code,
and are listed below.\\

$H_D$ = hydrostatic end force on area inside of flange \\
= $0.785B^2P$

$h_D$ = radial distance from the bolt circle, to the circle on which 
$H_D$ acts, as prescribed in Table 2-6 \\
= $(C-B)/2$

$H_T$ = difference between total hydrostatic end force and the hydrostatic 
end force on area inside of flange \\
= $H - H_D$

$h_T$ = radial distance from the bolt circle to the circle on which 
$H_T$ acts as prescribed in Table 2-6 \\ 
= $C/2 - (B/2+G/2)/2$

$H_G$ = gasket load for the operating condition \\
= $W_{m1} - H$

$h_G$ = radial distance from gasket load reaction to the bolt circle \\
= $(C-G)/2$ \\

Therefore, the total flange moment for the operating conditions
$M_{op} = 3918.78$~pound$\times$in.\\

For gasket seating, the total flange moment is calculated based on
Eq.~2-6(6) in the ASME code,

\begin{equation}
    M_a = W\frac{C-G}{2},
\end{equation}

where $M_a$ is denoted as $M_o$ in the ASME code.
$M_a = 4296.85$~pound$\times$in.\\

In Appendix~2-6 of the ASME code, it is stated,

\begin{displayquote}
    When the bolt spacing exceeds $2a + t$, multiply $M_O$ by the bolt spacing 
    correction factor $B_{SC}$ for calculating flange stress, where \\

    $B_{SC} = \sqrt{\frac{B_S}{2a+t}}$
\end{displayquote}

$B_{SC} = 2.28$ for the SLArPAAS I dewar.

% ----------------------------------------------------------------------
\subsection{Tangential Stress in Flange}
\label{app:tangential_stress}

The tangential stress, $S_T$, is defined in Appendix~2-7(b)(11) of the
ASME code,

\begin{equation}
    S_T = \frac{YM_o}{t^2B}.
\end{equation}


We calculate the operating conditions ($M_{op}$) with the bolt spacing 
factor, $B_{SC}$ applied,

\begin{equation}
    S_T = \frac{YM_{op}B_{SC}}{t^2B}.
\end{equation}

In addition, $Y$ is defined in Appendix~2-3 and Figure~2-7.1,

\begin{equation}
    Y = \frac{1}{K-1}[0.66845+5.7169\frac{K^2\log_{10}K}{K^2-1}],
\end{equation}

where $K = A/B$ is the ratio of outside diameter of flange to 
inside diameter of flange.

The tangential stress of the SLArPAAS I dewar flange is 12627.88~psi.
As the maximum allowable stress value in tension from applicable table
of stress values referenced by UG-23 (Table~ULT-23)
is 20~kpi = 20000~psi at temperature of 0, 100, 150$^{\circ}$F,
the tangential stress of the SLArPAAS I flange fulfils the requirement.
