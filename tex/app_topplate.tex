\section{Top Plate Design Pressure Calculations}
\label{app:top_plate}

The top plate of SLArpaas is made of stainless steel 304, and has
a thickness of 0.5~in, as shown in Fig.~\ref{fig:top_plate}.
All the calculations in this appendix can be found
at \url{https://github.com/yuntsebaryon/SLArpaasPressureAna/blob/main/notebook/Top%20Flange%20Stress.ipynb}.

\subsection{Minimum Required Top Plate Thickness}
\label{app:blind_flange}

The minimum required thickness for the top plate is calcuated assuming
a blind plate.  Eq. UG-34(c)(2)(1) below is used.

\begin{equation}
    t= d\sqrt{CP/SE}
\end{equation}

%-------------------------------------------------------------------
\begin{table}[h]
\begin{center}
\tabcolsep=10pt
\begin{tabular}{r|l|l|l}
\hline
\hline
Variable & Value & Unit & Reference \\
\hline
Diameter $d$ & 18 & in & design drawing \\
C & 0.25 & n/a & sketch (p) of Fig. UG-34 \\
P & 12.5 & psig & 10\% beyond the Cryofab spec. \\
S & 20000 & psi & Table ULT-23 at 0, 100, 150$^{\circ}$F \\
E & 1 & n/a & Table UW-12 \\
\hline
Minumum Required Thickness $t$ & 0.225 & in & UG-34 (c)(2)(1) \\
\hline
SLArpaas Top Plate Thickness & 0.5 & in & design drawing \\
\hline
\hline
\end{tabular}
\caption{Blind flange parameters to determine the minimum required thickness.}
\label{table:blind_flange}
\end{center}
\end{table}
%-------------------------------------------------------------------


The SLArpaas top plate has a thickness of 0.5~in, greater than the minimum
required thickness, 0.225~in.

% ----------------------------------------------------------------------
\subsection{Openings on the Top Plate}
\label{app:openings}

% ----------------------------------------------------------------------
\subsubsection{Welded Connections}
\label{app:welded}

The required minimum thickness of SLArpaas is 0.225~in, and therefore
UG-36(c)(3)(-a) applies,

\begin{displayquote}
    3 1/2 in. (89 mm) diameter -- in vessel shells or heads with a 
    required minimum thickness of 3/8~in. (10~mm) or less;
\end{displayquote}

The largest openings have the diameter of 3~in, 
and therefore no reinforcement is required for welded connections.

% ----------------------------------------------------------------------
\subsubsection{Reinforcement Required for Openings}
\label{app:opening_reinforcement}

The reinforcement requirements are based on UG-36(c)(3)(-c):

\begin{displayquote}
    no two isolated unreinforced openings, in accordance with (-a) 
    or (-b) above, shall have their centers closer to each other than 
    the sum of their diameters;
\end{displayquote}

Table~\ref{table:opening_dist} lists the distances of the openings on SLArpaas.

%-------------------------------------------------------------------
\begin{table}[h]
\begin{center}
\tabcolsep=10pt
\begin{tabular}{m{2cm}|m{2cm}|l|m{2cm}|m{2.5cm}}
\hline
\hline
Diameter of Opening 1 (in) & 
Diameter of Opening 2 (in) & Distance (in) & 
Required Distance (in) & 
Reinforcement Required? \\
\hline
3 & 3 & 8.113, 8.113, 9 & 6 & No \\
1.7 & 1.7 & 7.601 & 3.4 & No \\
1.2 & 1.2 & 2.253 & 2.4 & Yes (1) UG-39(b)(2) \\
3 & 1.2 & 3.536, 4.977 & 4.2 & Yes (2) UG-39(b)(2) \\
3 & 1.7 & 4.562 & 4.7 & Yes (3) UG-39 (b)(2) \\
3 & 1.7 & 4.707, 5.78 & 4.7 & No \\
\hline
\hline
\end{tabular}
\caption{Distances between the openings on the top plate of
SLArpaas, and the reinforcement requirements.}
\label{table:opening_dist}
\end{center}
\end{table}
%-------------------------------------------------------------------

The total cross-section area of the reinforcement is given by 
Eq.~UG-39(b)(1),

\begin{equation}
    A = 0.5dt + tt_n(1-f_{r1}),
\end{equation}

where $f_{r1}$ is defined in UG-37. 
As this case corresponds to sketch (o) in Figure~UG-40,
$f_{r1} = 1$.
Therefore the total cross-section area of the reinforcement becomes,

\begin{equation}
    A = 0.5dt,
\end{equation}
where $t$ is the minimum required thickness, 0.225~in,
and $d$ is finished diameter of circular opening or finished dimension 
(chord length at mid-surface of thickness excluding excess thickness 
available for reinforcement) of nonradial opening in the plane under 
consideration, in. (mm) (UG-37).

The reinforcement requirements can be found in UG-39(b)(2),

\begin{displayquote}
    Multiple openings none of which have diameters exceeding one‐half 
    the head diameter and no pair having an average diameter greater 
    than one‐quarter the head diameter may be reinforced individually 
    as required by (1) above when the spacing between any pair of 
    adjacent openings is equal to or greater than twice the average 
    diameter of the pair.\\

    When spacing between adjacent openings is less than twice but equal 
    to or more than 1 1/4 the average diameter of the pair, the required 
    reinforcement for each opening in the pair, as determined by (1) 
    above, shall be summed together and then distributed such that 50\% 
    of the sum is located between the two openings. 
    Spacings of less than 1 1/4 the average diameter of adjacent openings 
    shall be treated by rules of U-2(g).
\end{displayquote}

The existing reinforcement is greater than the shortest distance 
between the two openings times the extra top plate thickness
with respect to the minimum required thickness $t$,

\begin{equation}
    R = (s - \davg)\times(\tslarpaas-t).
\end{equation}

$s$ is the distance between the centers of the two openings,
$\davg$ is the average diameter of the two openings,
and $\tslarpaas$ is the actual thickness of the top plate.
Reinforcement is fulfiled if $R$ is greater than $0.5A$,
where $A = 0.5\davg t$ when the two openings have different 
diameters.
Table~\ref{table:opening_reinforcement} shows the parameters
and the calculation results of the openings requiring 
reinforcement.

%-------------------------------------------------------------------
\begin{table}[h]
\begin{center}
\tabcolsep=10pt
\begin{tabular}{l|m{2cm}|m{2cm}|m{1.4cm}|l|l|m{2.2cm}}
\hline
\hline
No. & Diameter of Opening 1 (in) & 
Diameter of Opening 2 (in) & 
Distance $s$ (in) &
$A$ (in$^2$) & $R$ (in$^2$) & Reinforcement Fulfiled? \\
\hline
(1) & 1.2 & 1.2 & 2.253 & 0.135 & 0.289575 & Yes \\
(2) & 3 & 1.2 & 3.536 & 0.23625 & 0.3949   & Yes \\
(3) & 3 & 1.7 & 4.562 & 0.264375 & 0.6083  & Yes \\
\hline
\hline
\end{tabular}
\caption{Calculation of required reinforcements.  The numbering 
corresponds to Table~\ref{table:opening_dist}. }
\label{table:opening_reinforcement}
\end{center}
\end{table}
%-------------------------------------------------------------------

In summary, the reinforcement requirements are satisfied for
all the openings.
