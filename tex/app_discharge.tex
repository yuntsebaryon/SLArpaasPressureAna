\section{Discharge Flowrate Calculation}
\label{app:discharge}

We calculate the the discharge capacity for Ar of the SLArpaas pressure relief device, 
the MDC precision burst disc 420033,
and assess the required discharge capacity in a few failure scenarios in this Appendix.
Details of the calculation can be found in 
\url{https://github.com/yuntsebaryon/SLArpaasPressureAna/blob/main/notebook/DischargeFlowrate.ipynb}.

\subsection{Discharge Capacity}
\label{app:discharge_capacity}

The MDC precision burst disc has a discharge capacity of 435~scfm for air.
The corresponding discharge capacity of Ar is obtained from ASME Mandatory Appendix 11-1,
\begin{equation}
    \label{eq:discharge}
    W = CKAP\sqrt{\frac{M}{T}},
\end{equation}
where $C$ is defined in ASME Fig.~11-1,
\begin{equation}
    \label{eq:constant_k}
    C = 520\sqrt{k\left(\frac{2}{k+1}\right)^{\frac{k+1}{k-1}}},
\end{equation}
and the rest of the parameter is defined in Table~\ref{table:discharge}.
$K$, $A$, $P$, $T$ are canceled out because they are identical for air and Ar
with the same burst disc,
\begin{equation}
    \frac{W_{Ar}}{W_{air}} = \frac{C_{Ar}\sqrt{M_{Ar}}}{C_{air}\sqrt{M_{air}}}.
\end{equation}

%-------------------------------------------------------------------
\begin{table}[h]
\begin{center}
\tabcolsep=10pt
\begin{tabular}{>{\raggedleft}m{5cm}|c|l|l|m{5.5cm}}
\hline
\hline
Variable & & Value & Unit & Reference \\
\hline
Actual discharge area of the safety valve & $A$ & & in$^2$ & \\
Constant for gas or vapor which is function of the ratio of specific heats, 
$k = C_p/C_V$ & $C$ & & n/a & ASME Figure 11-1 \\
   & $C_{air}$ & 356 & US unit & ASME App.~11-1 \\
   & $C_{Ar}$ & 377.83 & US unit & Eq.~\ref{eq:constant_k} \\
Ratio of specific heats for Ar & $k$ & 1.67 & n/a & $k = C_p/C_V$ \\
Heat capacity at constant pressure for Ar & $C_p$ & 20.834 & J/mol$\times$K & NIST database \\
Heat capacity at constant volume for Ar & $C_V$ & 12.479 & J/mol$\times$K & NIST database \\
Coefficient of discharge & $K$ & & n/a & UG-131(d) and UG-131(e) \\
Molecular weight & $M$ & & mol wt &  \\
   & $M_{air}$ & 28.97 & mol wt & ASME App.~11-1 \\
   & $M_{Ar}$ & 39.962 & mol wt & wiki \\
(Set pressure x 1.10) plus atmospheric pressure & $P$ & & psia & \\
Absolute temperature at inlet & $T$ & 60 & $^{\circ}$F & [($^{\circ}$F + 460) (K)], 
SCF definition \\
\hline
Flow of any gas or vapor, here for Ar & $W$ & 1.43 & lb/hr &  \\ 
Discharge capacity for Ar & & 390.9 & scfm & \\
\hline
\hline
\end{tabular}
\caption{Parameters to calculate the discharge capacity of Ar through the burst disc.}
\label{table:discharge}
\end{center}
\end{table}
%-------------------------------------------------------------------

The burst disc has the discharge capacity of 390.9~scfm for Ar.

% ------------------------------------------------------------------
\subsection{Initial LAr Filling}

At the beginning of the LAr filling, the LAr contacts the cryostat at room 
temperature and evaporates.
If we fail to open the venting valve V33, the SLArpaas vessel will be pressurized,
and the burst disc will crack at 10~psig.
Based on our experience, it takes at least one hour to consume a 230-L LAr
supply dewar -- in fact, it often takes 2--2.5~hours, and we starts with
a smaller opening.
Therefore, using 230/60~L/min is an overestimate of the LAr flow at the
initial filling. \\

The overestimated Ar gas amount produced at the initial LAr filling
is 3293.5~slpm, or 115.3~scfm, around 1/3 of the burst disc discharge capacity.

% ------------------------------------------------------------------
\subsection{Operation Failure Scenarios}

There are two main operation failure scenarios:
\begin{enumerate}
    \item thermosyphon failure, losing its cooling power,
    \item loss of the vacuum in the insulating jacket around the SLArpaas vessel, 
    allowing air or Ar filling in.
\end{enumerate}
We calculate the individual heat load owing to the failure scenarios,
and evaluate the Ar gas per minute produced by evaporation from this heat load,
thereby verify the discharge capacity requirement of the burst disc.\\

The heat transfer per unit time (power) is calculated by the equation,
\begin{equation}
    \label{eq:heat_transfer}
    P = k\frac{A}{L}(T_H-T_L),
\end{equation}
where $k$ is the thermal conductivity of the material, $A$ is the area where
the heat is tranferred across, $L$ is the thickness or the length of the 
heat tranfer, and $T_H$ and $T_L$ are the high and low temperatures, respectively.
All the parameters, together with the latent heat of Ar evaporation,
and the results
are listed in Table~\ref{table:failure_heat}.

%-------------------------------------------------------------------
\begin{table}[h]
\begin{center}
\tabcolsep=10pt
\begin{tabular}{>{\raggedleft}m{3.5cm}|c|l|l|m{6.5cm}}
\hline
\hline
Variable & & Value & Unit & Reference \\
\hline
Stainless steel 304 conductivity at 100~K & $k_{SS}$ & 9.34 & W/mK & 
Thermal Conductivity by T. Ashworth and D. Smith \\
Air conductivity at 293~K & $k_{air}$ & 0.02556 & W/mK & 
\url{https://www.engineeringtoolbox.com/air-properties-viscosity-conductivity-heat-capacity-d_1509.html} \\
Ar conductivity at 293~K & $k_{Ar}$ & 0.0172 & W/mK &
\url{https://www.engineeringtoolbox.com/argon-d_1414.html} \\
Ar Latent Heat of Evaporation at boiling point &  & 163000 & J/kg & 
\url{https://www.engineeringtoolbox.com/argon-d_1414.html} \\
Ar volume at 87.305~K and 14.7~psia (boiling point) &  & 0.17316 & \unit{\cubic\m}/kg & NIST database \\
Ar volume at 293~K and 14.7~psia (STP) &  & 0.60127 & \unit{\cubic\m}/kg & NIST database \\
SLArPAAS I vacuum jacket thickness -- wall & $L_w$ & 2.54 & cm & Fig.~\ref{fig:dewar} \\
SLArPAAS I vacuum jacket surface area -- wall & $A_w$ & 6810 & \unit{\cm\squared} & Fig.~\ref{fig:dewar} and 
\url{https://github.com/yuntsebaryon/SLArpaasPressureAna/blob/main/notebook/DischargeFlowrate.ipynb} \\
SLArPAAS I vacuum jacket thickness -- bottom & $L_b$ & 6.38 & cm & Fig.~\ref{fig:dewar} and 
\url{https://github.com/yuntsebaryon/SLArpaasPressureAna/blob/main/notebook/DischargeFlowrate.ipynb} \\
SLArPAAS I vacuum jacket surface area -- bottom & $A_b$ & 1064 & \unit{\cm\squared} & Fig.~\ref{fig:dewar} and 
\url{https://github.com/yuntsebaryon/SLArpaasPressureAna/blob/main/notebook/DischargeFlowrate.ipynb} \\
\hline
\multicolumn{5}{l}{The vacuum jacket of the SLArPAAS I cryostat is full of air} \\
Heat load &  & 148.85 & W & \\
Ar evaporated by the heat load & & 1.17 & scfm & \\
\hline
\multicolumn{5}{l}{The vacuum jacket of the SLArPAAS I cryostat is full of Ar} \\ 
Heat load &  &  100.16 & W & \\
Ar evaporated by the heat load & & 0.78 & scfm & \\
\hline
\hline
\end{tabular}
\caption{Parameters to calculate the head load and the evaporated Ar with a few operation failure 
scenarios and the results.}
\label{table:failure_heat}
\end{center}
\end{table}
%-------------------------------------------------------------------

% ------------------------------------------------------------------
\subsubsection{Thermosyphon Failure}

With the geometry of the thermosyphon evaporator used in SLArpaas not finalized,
we don't have the exact number of its cooling power.
However, we estimate the head load of the SLArpaas system is about $\sim$100~W,
and based on the past calculation and experience, each thermosyphon evaporator
can provide 100 -- 200~W.
We therefore use an overestimate cooling power in this case: if 1000~W
of heat load will evaporate the amount of Ar that the burst disc discharge capacity
cannot cope with?
The calculation indicates the Ar flow from the 1000~W of heat will be 7.82~scfm
and the discharge rate of the burst disc is not a problem.

% ------------------------------------------------------------------
\subsubsection{SLArpaas Vacuum Insulation Failure}

We consider the two scenarios when the vacuum insulation jacket around the
SLArpaas vessel fails: air and Ar filled in.
The upper bound of the heat transfer is the case where the vacuum jacket
is full of air or Ar at room temperature and 1~atm.
Effects from the superinsulation layers inside the vacuum jacket are negligible.
The Ar flow is 1.17 and 0.78~scfm in the air and Ar cases, respectively,
and the discharge capacity of the burst disc exceeds the requirement.
