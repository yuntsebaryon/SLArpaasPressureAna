\section{Dewar Buckling Analysis}
\label{app:buckling}

The minimum unsupported length beyond which the critical pressure, $P_c$, is independent of $L$
is called the critical length $L_c$ and is expressed as
\begin{equation}
    L_c = 1.14(1-\nu^2)^{1/4}D_0(D_0/t)^{1/2},
\end{equation}
and for $\nu=0.3$,
\begin{equation}
    \label{eq:critical_l}
    L_c = 1.11D_0(D_0/t)^{1/2}.
\end{equation}
All the variables are described in Table~\ref{table:buckling}, and, 
the Poisson's ratio, $\nu= 0.3$, together with all the formulas used in the Cryofab analysis,
are taken from \textit{Pressure Vessel Design Handbook}.
The choice of the Poisson's ratio can be found in the wiki,
``Most steels and rigid polymers when used within their design limits (before yield) 
exhibit values of about 0.3, increasing to 0.5 for post-yield deformation which occurs 
largely at constant volume.''

%-------------------------------------------------------------------
\begin{table}[h]
\begin{center}
\tabcolsep=10pt
\begin{tabular}{>{\raggedleft}m{4.5cm}|c|l|l|m{5.5cm}}
\hline
\hline
Variable & & Value & Unit & Reference \\
\hline
Inner wall thickness & $t$ & 0.024 & inch & Cryofab analysis \\
Inner cylinder effective length & $L$ & 24.92 & inch & Cryofab analysis \\
Inner diameter & $D_0$ & 14 & inch & Cryofab analysis \\
Poisson's ratio & $\nu$ & 0.3 & n/a & Pressure Vessel 
Design Handbook\tablefootnote{Henry H. Bednar, P.E., Second Edition (1986), p.50--51} 
and Wiki\tablefootnote{\url{https://en.wikipedia.org/wiki/Poisson's_ratio}} \\
Elastic modulus for SS304 & $E$ & $28\times 10^6$ & psi & AISI 304 Stainless 
Steel\tablefootnote{\url{https://304stainlesssteel.org/304-stainless-steel-properties/}} \\
\hline
Critical length for the inner shell & $L_c$ & 375 & inch & Eq.~\ref{eq:critical_l} \\
Critical pressure for the inner shell & $P_c$ & 4.98 & psi & Eq.~\ref{eq:critical_P} \\
\hline
\hline
Outer wall thickness & & 0.060 & inch & Cryofab analysis \\
Outer cylinder effective length & & 25.70 & inch & Cryofab analysis \\
Outer diameter & & 16 & inch & Cryofab analysis \\
\hline
Critical length for the outer shell & & 290 & inch & Eq.~\ref{eq:critical_l} \\
Critical pressure for the outer shell & & 39 & psi & Eq.~\ref{eq:critical_P} \\
\hline
\hline
\end{tabular}
\caption{Buckling analysis for the dewar}
\label{table:buckling}
\end{center}
\end{table}
%-------------------------------------------------------------------


For cylindrical shells with $L<L_c$,
Cryofab uses the US Experimental Model Basin formula to calculate the critical pressure, $P_c$,
under uniform external pressure,
\begin{equation}
    P_c = \frac{2.42E}{(1-\nu^2)^{3/4}}\left[\frac{(t/D_0)^{2.5}}{(L/D_0)-0.45(t/D_0)^{0.5}}\right].
\end{equation}
With $\nu = 0.3$ and $0.45(t/D_0)^{0.5}$ being negligible, $P_c$ can be written as
\begin{equation}
    \label{eq:critical_P}
    P_c = 2.6E(t/D_0)^{2.5}/(L/D_0).
\end{equation}

The results are presented in Table~\ref{table:buckling}.
Collapse occurs when 4.98~psi is applied outside the inner vessel shell; in other words,
when the vacuum jacket fails and has 4.98~psi relative to the pressure in the inner vessel.

%
% ASME code calculation
%
% I use was UG-28, (c)(1) Step 7, with A = 0.000065 (obtained from (c)(1)),
% and the result of the maximum allowable external working pressure, P_a = 2.08 psi
%