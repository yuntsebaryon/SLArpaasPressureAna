\section{Dewar Pressure Analysis}
\label{app:dewar}

The Cryofab CF 1424-F dewar is made of stainless steel 304,
as shown in Fig.~\ref{fig:dewar}.
All the calculations in this appendix can be found at
\url{https://github.com/yuntsebaryon/SLArpaasPressureAna/blob/main/notebook/DewarPressure.ipynb}.

% ---------------------------------------------------------
\subsection{Shells under Internal Pressure}
\label{app:shell}

The Cryofab pressure analysis only calculates the longitudinal 
stress for the shell, based on UG-27(c)(2):

\begin{displayquote}
    Longitudinal Stress (Circumferential Joints). When the thickness does not exceed 
    one‐half of the inside radius, or $P$ does not exceed 1.25$SE$ , 
    the following formulas shall apply:\\
\end{displayquote}

\begin{equation}
    P = \frac{2SEt}{R-0.4t}
\end{equation}
\\

The parameters of the dewar and the resulting shell longitudinal stress
can be found in Table~\ref{table:dewar}.

%-------------------------------------------------------------------
\begin{table}[h]
\begin{center}
\tabcolsep=10pt
\begin{tabular}{>{\raggedleft}m{4.5cm}|c|l|l|m{5.5cm}}
\hline
\hline
Variable &  & Value & Unit & Reference \\
\hline
Internal design pressure & $P$ & 12.5 & psig & 10\% beyond the Cryofab spec. \\
Maximum allowable stress value & $S$ & 20000 & psi & Table ULT-23 at 0, 100, 150$^{\circ}$F \\
Joint efficiency & $E$ & 0.7 & n/a & Cryofab pressure analysis \\
Inside radius of the shell course & $R$ & 7 & in & Cryofab \\
Inside spherical or crown radius & $L$ & 14 & in & Cryofab pressure analysis uses 
the inner diameter \\
Shell thickness & $t$ & 0.024 & in & Cryofab pressure analysis \\
Bottom thickness & $t_b$ & 0.048 & in & Cryofab pressure analysis \\
\hline
Longitudinal stress & n/a & 96.13 & psi &  UG-27(c)(2)(2) \\
Torespherical bottom stress & n/a & 46.08 & psi & UG-32(d)(2) \\
\hline
\hline
\end{tabular}
\caption{Dewar parameters to determine the longitudinal shell stress
and the torispherical bottom stress.}
\label{table:dewar}
\end{center}
\end{table}
%-------------------------------------------------------------------


% ---------------------------------------------------------
\subsection{Torispherical Bottom under Internal Pressure}
\label{app:bottom}

The bottom of the dewar has a torispherical shape based on the Cryofab
analysis.
Eq.~UG-32(d)(2) is therefore applied,
\begin{equation}
    P = \frac{SEt}{0.885L+0.1t},
\label{eq:UG-32_d_2}
\end{equation}

Also, the bottom will be at the liquid argon temperature, 87~K,
and according to UG-32(d)(2),

\begin{displayquote}
    Torispherical heads made of materials having a specified minimum 
    tensile strength exceeding 70,000 psi (485 MPa) shall be designed 
    using a value of S equal to 20,000 psi (138 MPa) at room temperature 
    and reduced in proportion to the reduction in maximum allowable 
    stress values at temperature for the material (see UG-23).
\end{displayquote}

As per Table~ULT-23, at the room temperature, the maximum allowable 
stress, $S$, is 20000 psi, 
while $S = 23400$~psi at $-300^{\circ}$F (88.71~K, the LAr temperature). 
Therefore the reduction rate is $20000/23400 = 0.85$.
(0.85 is used in the Cryofab analysis.
I am wondering if it is correct -- should it be actually 23400/20000?)\\

The thickness, $t$, in Eq.~\ref{eq:UG-32_d_2} should be the thickness
of the bottom, $t_b$ in Table~\ref{table:dewar}, and the maximum allowable
pressure can be evaluated by
\begin{equation}
    P = \frac{0.85SEt_b}{0.885L+0.1t_b} = 46.08~\text{psi}
\end{equation}

% ---------------------------------------------------------
\subsection{Maximum Allowable Pressure in the Dewar}
\label{app:mawp}

The maximum allowable pressure in the dewar is the minimum value of
the results in App.~\ref{app:shell} and \ref{app:bottom}, 46.08~psi,
which is larger than the designate maximum allowable working pressure,
10~psig.
The pressure requirements are satisfied.