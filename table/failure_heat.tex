%-------------------------------------------------------------------
\begin{table}[h]
\begin{center}
\tabcolsep=10pt
\begin{tabular}{>{\raggedleft}m{3.5cm}|c|l|l|m{6.5cm}}
\hline
\hline
Variable & & Value & Unit & Reference \\
\hline
Stainless steel 304 conductivity at 100~K & $k_{SS}$ & 9.34 & W/mK & 
Thermal Conductivity by T. Ashworth and D. Smith \\
Air conductivity at 293~K & $k_{air}$ & 0.02556 & W/mK & 
\url{https://www.engineeringtoolbox.com/air-properties-viscosity-conductivity-heat-capacity-d_1509.html} \\
Ar conductivity at 293~K & $k_{Ar}$ & 0.0172 & W/mK &
\url{https://www.engineeringtoolbox.com/argon-d_1414.html} \\
Ar Latent Heat of Evaporation at boiling point &  & 163000 & J/kg & 
\url{https://www.engineeringtoolbox.com/argon-d_1414.html} \\
Ar volume at 87.305~K and 14.7~psia (boiling point) &  & 0.17316 & \unit{\cubic\m}/kg & NIST database \\
Ar volume at 293~K and 14.7~psia (STP) &  & 0.60127 & \unit{\cubic\m}/kg & NIST database \\
SLArpaas vacuum jacket thickness -- wall & $L_w$ & 2.54 & cm & Fig.~\ref{fig:dewar} \\
SLArpaas vacuum jacket surface area -- wall & $A_w$ & 6810 & \unit{\cm\squared} & Fig.~\ref{fig:dewar} and 
\url{https://github.com/yuntsebaryon/SLArpaasPressureAna/blob/main/notebook/DischargeFlowrate.ipynb} \\
SLArpaas vacuum jacket thickness -- bottom & $L_b$ & 6.38 & cm & Fig.~\ref{fig:dewar} and 
\url{https://github.com/yuntsebaryon/SLArpaasPressureAna/blob/main/notebook/DischargeFlowrate.ipynb} \\
SLArpaas vacuum jacket surface area -- bottom & $A_b$ & 1064 & \unit{\cm\squared} & Fig.~\ref{fig:dewar} and 
\url{https://github.com/yuntsebaryon/SLArpaasPressureAna/blob/main/notebook/DischargeFlowrate.ipynb} \\
\hline
\multicolumn{5}{l}{The vacuum jacket of the SLArpaas cryostat is full of air} \\
Heat load &  & 148.85 & W & \\
Ar evaporated by the heat load & & 1.17 & scfm & \\
\hline
\multicolumn{5}{l}{The vacuum jacket of the SLArpaas cryostat is full of Ar} \\ 
Heat load &  &  100.16 & W & \\
Ar evaporated by the heat load & & 0.78 & scfm & \\
\hline
\hline
\end{tabular}
\caption{Parameters to calculate the head load and the evaporated Ar with a few operation failure 
scenarios and the results.}
\label{table:failure_heat}
\end{center}
\end{table}
%-------------------------------------------------------------------